% !TEX encoding = utf8
% !TeX spellcheck = pl_PL


\documentclass[a4paper, 10pt]{article}
\usepackage[utf8]{inputenc}
\usepackage[polish]{babel}
\usepackage{polski}
\usepackage{graphicx}
\usepackage{listings}
\usepackage{amsfonts}
\usepackage{amsmath}
\usepackage{geometry}

\usepackage{float}


\author{Jakub Postępski}
\title{STP - Projekt 2.24}
\graphicspath{{../images/}}
\newgeometry{tmargin=3cm, bmargin=0.5cm, lmargin=0.5cm, rmargin=0.5cm}

\begin{document}
	\maketitle
	\section{Wyznaczanie modeli rekurencyjnych}
	Dla podanych danych wyznaczono modele rekurencyjne (tab. \ref{tab:z1}). Wybrano najlepszy model dla $\tau=6$. Na początku przebiegu modele wykazują mniejszy błąd wyjścia. \\
	\begin{table}[H]
	\begin{tabular}{|c|c|c|c|c|c|c|}
	\hline 
	$\tau$ & $b_\tau$ & $b_{\tau-1}$ & $-a_1$ & $-a_2$ & $E$ & rys. \\ 
	\hline 
	 1 & -0.0082 & -0.0679 &  1.6619 & -0.6799
	  & 79.2282 & \\ 
	\hline 
	2 & 0.0027 & -0.1003
	  &  1.5901 & -0.6135 & 54.2224 & \\ 
	\hline 3 & -0.0014 & -0.1337 &  1.4685  & -0.5010 & 39.9572 & \\ 
	\hline 
	4 & -0.0568 & -0.1271 & 1.3107 & -0.3552
	  & 32.3080  &\\ 
	\hline 
 5 & -0.1188 & -0.1281 & 1.1146 &  -0.1763 & 20.9694 & \\ 
	\hline 
	6 & -0.1854  &  -0.0918 &  1.0403  & -0.1135 & 16.3988 & \\ 
	\hline 
	7 & -0.1575 & -0.0517  & 1.2836 & -0.3445  & 46.8691
	& \\ 
	\hline
	8 & -0.1146 & 0.0001  & 1.5764 & -0.6144 & 176.8823 & \\ 
	\hline 
	\end{tabular}
	\label{tab:z1}
	\caption{Porównanie modeli rekurencyjnych}
	\end{table}
	
	\begin{figure}[H]
	\centering
	\includegraphics[width=0.9\linewidth]{z1_1}
	\caption{Wyjścia modelu dla $\tau=1$}
	\label{fig:z1_1}
	\end{figure}
	\begin{figure}[H]
	\centering
	\includegraphics[width=0.9\linewidth]{z1_2}
	\caption{Wyjścia modelu dla $\tau=2$}
	\label{fig:z1_2}
	\end{figure}
	\begin{figure}[H]
	\centering
	\includegraphics[width=0.9\linewidth]{z1_3}
	\caption{Wyjścia modelu dla $\tau=3$}
	\label{fig:z1_3}
	\end{figure}
	\begin{figure}[H]
	\centering
	\includegraphics[width=0.9\linewidth]{z1_4}
	\caption{Wyjścia modelu dla $\tau=4$}
	\label{fig:z1_4}
	\end{figure}
	\begin{figure}[H]
	\centering
	\includegraphics[width=0.9\linewidth]{z1_5}
	\caption{Wyjścia modelu dla $\tau=5$}
	\label{fig:z1_5}
	\end{figure}
	\begin{figure}[H]
	\centering
	\includegraphics[width=0.9\linewidth]{z1_6}
	\caption{Wyjścia modelu dla $\tau=6$}
	\label{fig:z1_6}
	\end{figure}
	\begin{figure}[H]
	\centering
	\includegraphics[width=0.9\linewidth]{z1_7}
	\caption{Wyjścia modelu dla $\tau=7$}
	\label{fig:z1_7}
	\end{figure}
	\begin{figure}[H]
	\centering
	\includegraphics[width=0.9\linewidth]{z1_8}
	\caption{Wyjścia modelu dla $\tau=8$}
	\label{fig:z1_8}
	\end{figure}


\end{document}